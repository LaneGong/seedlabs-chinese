In this lab, the attacker needs to be able to sniff packets,
but running sniffer programs inside a container has problems, because
a container is effectively attached to a virtual switch, 
so it can only see its own traffic, and it is never going to see 
the packets among other containers. To solve this problem,
we use the \texttt{host} mode for the attacker container. This
allows the attacker container to see all the traffics. The following
entry used on the attacker container:

\begin{lstlisting}
network_mode: host
\end{lstlisting}

When a container is in the \texttt{host} mode,  it sees
all the host's network interfaces, and it even has the same
IP addresses as the host. Basically, it is put in the
same network namespace as the host VM. However, the container
is still a separate machine, because its other namespaces are
still different from the host.

